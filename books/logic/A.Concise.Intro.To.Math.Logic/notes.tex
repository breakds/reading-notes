\documentclass[a4paper, 12pt]{article}
\title{Notes: A Concise Introduction to Mathematical Logic}
\author{BreakDS}
\date{}

\usepackage{amssymb}
\usepackage{amsmath}
\usepackage{graphicx}
\usepackage[pagebackref=true,breaklinks=true,letterpaper=true,colorlinks,bookmarks=false]{hyperref}


\begin{document}
\maketitle

\section{Propositional Loigc}
\subsection{Boolean Function and Formulas}
\begin{enumerate}
\item There are some useful definitions.
  \begin{itemize}
  \item There are strings called \textbf{atomic} strings, which cannot
    be divided further in this \textbf{formal language}. They are also
    called \textbf{prime}s.
  \item \textbf{prime}s are usually represented by $p$.
  \end{itemize}
\end{enumerate}
\subsection{}
\subsection{}
\subsection{A Calculus of Natural Deduction}
\begin{enumerate}
\item[4.2] Note: A not-so-obvious conclusion is that $X \not \vdash
  \alpha \Rightarrow X, \neg \alpha \not \vdash \bot$. It seems a
  little bit counter-intuitive but it should reads:


  If $X \not \vdash \alpha$, then of course $X \not \vdash \text{
    everything}$, therefore by definition, $X$ must be
  consistent. However, we can actually get more: even $X, \neg \alpha$
  is consistent.

  
\item[4.3] Lindenbaum's Theorem (a.k.a Compactness)
\item[4.5] This lemma should read as - A maximally consistent set X is
  \textbf{yet still} satisfiable. The idea behind this proof is that
  we would like to prove statisfiability $\Leftrightarrow$
  consistency, and to be sepcific consistency $\Rightarrow$
  satisfiability. In order to prove that, we prove a stricter version
  of the statementm, which is maximally consistency $\Rightarrow$
  satisfiability.

  Some notes on the proof:
  \begin{itemize}
  \item Define a valuation $\omega$, so that $\omega \models p,
    \forall \text{ prime } p \in X$, and $\omega \not \models p,
    \forall \text{ prime } p \not \in X$.
  \item Note that $\alpha \in X$ is equivalent to $X \vdash \alpha$,
    if $X$ is maximally satisfiable.
  \item Try to prove that $\forall \alpha \in X, \omega \models \alpha$.
  \end{itemize}

\item[4.6] Proof of this theorem should be quite straight forward and
  the key is that 

  
\end{enumerate}

\subsection{Application of the Compactness Theorem}

First, some notes:
\begin{itemize}
\item The Compactness Theorem is crucial here as it extends proof of
  some finite sets to an (potentially) infinite set.
\item That means to form those proofs, we often need to prove the
  finite set cases first.
\item And inorder to apply the Compactness Theorem, we need to write
  binary variable (or, propositional variable) form equations
  (formulas) that needs to be true.
\item That is why we would like to use the Compactness Theorem of
  $\models$ rather than $\vdash$. We need \textbf{valuation}.
\end{itemize}


\end{document}