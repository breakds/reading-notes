\documentclass[a4paper, 12pt]{article}
\title{Exercise: A Concise Introduction to Mathematical Logic}
\author{BreakDS}
\date{}

\usepackage{amssymb}
\usepackage{amsmath}
\usepackage{graphicx}
\usepackage[pagebackref=true,breaklinks=true,letterpaper=true,colorlinks,bookmarks=false]{hyperref}


\begin{document}
\maketitle

\section{Propositional Loigc}
\subsection{Boolean Function and Formulas}
\subsection{}
\subsection{}
\subsection{A Calculus of Natural Deduction}
\begin{enumerate}
\item The proof should be straight forward: By Theorem 4.1, The in
  consistency of $X \cup \{ \neg \alpha | \alpha \in Y\}$ has a finite
  subset $H$ that is inconsistent, namely $H \vdash \bot$. We can
  write $H$ as 
  \[
  H =  \neg \alpha_0 \cup \neg \alpha_1 \cup \neg \alpha_2 \cdots \neg \alpha_n \cup X_0
  \]
  Where $\alpha_0, \alpha_1, \cdots, \alpha_n \in Y$, and $X_0$ is a
  finite subset of $X$. This is achievable because $H$ is finite.

  
  There are two cases, depending on whether $n = 0$.


  \begin{itemize}
  \item[\textbf{Case 1}] $n = 0$. This means that $X_0 \vdash \bot$,
    which means that $X_0$ is inconsistent, and by the definition of
    inconsistency $X_0 \vdash$ everything, including an arbitray
    $\alpha_0$ in $Y$. Proved.

  \item[\textbf{Case 2}] $n > 0$. Thie means:
    \[
    X_0 , \neg \alpha_0 , \neg \alpha_1 , \neg \alpha_2 \cdots \neg \alpha_n
    \vdash \bot
    \]
    Which by Lemma 4.2 shows 
    \[
    X_0 , \neg \alpha_0 , \neg \alpha_1 , \neg \alpha_2 \cdots \neg \alpha_{n-1}
    \vdash \alpha_n
    \]
    Which by $\rightarrow$-introduction shows 
    \[
    X_0 , \neg \alpha_0 , \neg \alpha_1 , \neg \alpha_2 \cdots \neg \alpha_{n-2}
    \vdash \neg \alpha_{n-1} \rightarrow \alpha_n
    \]
    And by definition of $\rightarrow$, we have 
    \[
    X_0 , \neg \alpha_0 , \neg \alpha_1 , \neg \alpha_2 \cdots \neg \alpha_{n-2}
    \vdash \neg(\neg \alpha_{n-1} \wedge \neg \alpha_n)
    \]
    Namely
    And by definition of $\rightarrow$, we have 
    \[
    X_0 , \neg \alpha_0 , \neg \alpha_1 , \neg \alpha_2 \cdots \neg \alpha_{n-2}
    \vdash \alpha_{n-1} \vee \alpha_n
    \]
    Repeat the above deduction, we get
    \[
    X_0 \vdash \alpha_0 \vee \cdots \vee \alpha_{n-1} \vee \alpha_n
    \]
    And by basic rules of $\vdash$, since $X_0 \subseteq X$, $X$ does
    that as well.

  \end{itemize}


  
  
\end{enumerate}



\end{document}